\documentclass[]{jsarticle}
\usepackage[dvipdfmx]{graphicx}
\usepackage{comment}

\title{ネットワークプログラミング課題レポート}
\author{Ec5 \ 24番 \ 平田 蓮}
\date{}

\begin{document}
\maketitle

\section{ロボットの行動原理}
    今回作成したプログラムは、操作する船が場にいる船の中で最も多くのタンクを取得することを目標とする。
    そのため、「できる限り多くタンクを取得する」ことよりも、「他の船よりも多くタンクを取得する」
    ことに重点を置いてプログラムを設計した。

    具体的には、「他の船が取れないタンクに向かう」ことを行動原理とした。
    本課題においてプログラムを作成する場合、
    最も単純かつ簡単なプログラムは、「最も近いタンクに向かう」という行動原理に基づくものである。
    しかし、これでは、目標のタンクと自分の船の間に他の船が存在した場合、
    高確率で間の船にタンクを先に取得されてしまう。
    そのため、先述の「他の船が取れないタンクに向かう」ことで、
    他の船に邪魔されることなくタンクを取得できると考えた。

\section{行動原理に基づいたアルゴリズム}
    先述の行動原理に基づいたアルゴリズムを示す。

    \begin{enumerate}
        \item 各タンクについて、それから最も近い船を計算
        \item 最も近い船が自分の船であるタンクについて、最も得点が高いものを選択
        \item もし条件に当てはまるタンクが複数ある場合、自分の船が最も近いタンクを選択
        \item 選択したタンクに向かう
    \end{enumerate}

    先述の「最も近いタンクに向かう」という行動原理を実装する際には、
    自分の船と各タンクの距離を計算するが、このアルゴリズムは逆に、
    タンクと各船の距離を計算することで先述の行動原理を実現している。
    このアルゴリズムの特長として、
    「他の船に比べて自分の船が最も近いタンクを計算」した後、
    選択したタンクを自分の船、他のタンクを他の船と見立てて同様の計算を行った時、
    今選択したタンクに向かった後に向かうタンクについても計算を行うことができ、
    いわゆる「先読み」が簡単に実装できる点が挙げられる。
    (以下、これらをそれぞれ1階読み、2階読み等と記述する。)
    
\section{他アルゴリズムとの比較}
    プログラムを実装した後、
    実際に他のプログラムに比べてどれだけ優位に動くか調べるために、
    今回は、先で最も単純と述べた、「最も近いタンクに向かう」プログラム(これを簡易ロボットと呼ぶ)を作成し、
    得点の比較を行った。

    3種類のプログラムを2艇ずつ、計6艇の船が参加し、1秒ごとにタンクが浮上する
    3分間の試合を100回行い、各プログラムについて得点の平均、標準偏差を算出した。
    表に、結果を示す。

    \begin{table}[ht]
        \centering
        \begin{tabular}{c|ccc}
            & 簡易ロボット & 1階読み & 2階読み \\ \hline
            平均 & 103.8 & 189.5 & 210.4 \\
            標準偏差 & 23.4 & 10.3 & 18.9 \\
        \end{tabular}
    \end{table}

    表から分かるように、先述のアルゴリズムを用いると、より高得点を出すことがわかった。
    また、簡易ロボットは船の初期配置に左右されやすく、得点の分散が他より大きくなったことが予想される。

    先述のアルゴリズムが優位であったが、階数を増やすと、得点の分散が大きくなってしまい、
    不安定なプログラムになってしまうため、最終的には1階読みのプログラムを提出する。
\end{document}